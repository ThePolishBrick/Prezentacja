\documentclass{beamer}
\usepackage[T1]{fontenc}
\usetheme{Goettingen}
 
\title{Wielka wojna emu}
\titlegraphic{
    \includegraphics[width=2cm]{Emu.png}
}
\subtitle{Czyli jak australijskie wojsko przegralo z ptakami}
\author{Marek Jelinski}
\date{16.12.2024}

\begin{document}

\begin{frame}
    \titlepage
\end{frame}

\begin{frame}{Spis tresci}
    \tableofcontents
\end{frame}

\section{Sytuacja}
\section{Wojna}
    \subsection{Pierwszy okres}
    \subsection{Drugi okres}    
\section{Echa akcji}
\section{Bibliografia}

\begin{frame}
\frametitle{Sytuacja}
Po I wojnie swiatowej duza liczba bylych zolnierzy z Australii, wraz z szeregiem brytyjskich weteranow, zaczela uprawiac ziemie w Australii Zachodniej[1], czesto na obszarach marginalnych, wczesniej niezasiedlanych. Wraz z nadejsciem wielkiego kryzysu w 1929, rolnicy zostali zacheceni do zwiekszenia zbiorow pszenicy, przy czym rzad w Canberze obiecal dostarczyc pomoc w postaci dotacji dla rolnikow, czego nie zrealizowal. Pomimo zalecen i obiecanych dotacji, ceny pszenicy nadal spadaly, a do pazdziernika 1932 sytuacja stawala sie coraz powazniejsza: rolnicy przygotowywali sie wiec do zbioru plonow, jednoczesnie grozac odmowa dostarczenia pszenicy[2].
\end{frame}
\begin{frame}
Trudnosci, przed ktorymi staneli rolnicy, wzrosly wraz z rozrostem populacji emu w tym rejonie do 20 000 ptakow. Emu co prawda regularnie migrowaly po sezonie legowym, kierujac sie na wybrzeze z obszarow srodladowych i nie bylo to nic dziwnego. Po przygotowaniu ziemi pod pola, zorganizowaniu zasobow wodnych na niegoscinnych terenach Australii Zachodniej, emu zorientowaly sie jednak, ze tereny uprawne sa swietnym miejscem do zerowania i uzupelniania zapasow wody. W zwiazku z tym zaczely wkraczac na teren farm – w szczegolnosci w poblizu miejscowosci Chandler i Walgoolan. Wielka liczba ptakow pozerala i niszczyla plony, a takze pozostawiala duze szczeliny w pobudowanych plotach, co wykorzystywaly kroliki (stanowiace plage), ktore mogly wejsc na niedostepne dla nich dotad ziemie i powodowac dalsze problemy[3].
\end{frame}
\begin{frame}
Rolnicy obawiali sie, ze ptaki spustosza ich plony. W zwiazku z tym delegacja bylych zolnierzy, a obecnych farmerow i rolnikow, zostala wyslana na spotkanie z ministrem obrony sir George'em Pearce'em. Po sluzbie podczas I wojny swiatowej zolnierze-osadnicy byli swiadomi skutecznosci karabinow maszynowych i poprosili o ich rozmieszczenie w celu odstrzalu ptakow. Minister ostatecznie zgodzil sie na to, ale miejscowi musieli spelnic kilka warunkow: bron miala byc uzywana tylko przez zolnierzy, transport ludzi mial byc finansowany przez wladze Australii Zachodniej, a rolnicy mieli dostarczac wojsku zywnosc, zakwaterowanie i zaplacic za amunicje. 
\end{frame}
\begin{frame}
Motywacja dla ministra Pearce'a bylo to, ze ptaki bylyby – w jego mniemaniu – dobrym celem dla zolnierzy, ktorzy w ten sposob mieli cwiczyc strzelanie do duzej liczby szybkich celow. Liczono takze, ze bedzie to pewna forma obiecywanej wczesniej rzadowej pomocy dla rolnikow (niezrealizowane dotacje) i aby wykorzystac to propagandowo, wraz z zolnierzami wyslano operatora z Fox Movietone, agencji krecacej kroniki filmowe.

\includegraphics[width=5cm]{Emu2.png}
\end{frame}

\begin{frame}
\frametitle{Wojna}
Akcja wojska miała rozpocząć się początkowo w październiku 1932. Na dowódcę wyznaczono majora G.P.W. Mereditha z 7. ciężkiej baterii Królewskiej Artylerii Australijskiej. W skład oddziału weszli: sierżant S. McMurray i strzelec J. O’Hallora, uzbrojeni w dwa karabiny typu Lewis i 10 tysięcy sztuk amunicji. Operacja została jednak opóźniona przez porę deszczową, która spowodowała rozproszenie stad emu na rozleglejszym obszarze. Deszcz przestał padać dopiero 2 listopada 1932. Wówczas rozmieszczono wyznaczony oddział do walki z emu. Dodatkowo nakazano zebranie 100 ptasich skór, aby ich pióra mogły być użyte do wykonania czapek dla jeźdźców lekkiej kawalerii[4].
\end{frame}
\begin{frame}
\frametitle{Pierwszy okres}
Jeszcze 2 listopada żołnierze udali się do osady Campion, gdzie widziano około 50 emu. Ponieważ ptaki znajdowały się poza zasięgiem karabinów, lokalni osadnicy próbowali zapędzić ptaki w zasadzkę, ale ptaki podzieliły się na małe grupy i uciekały tak sprawnie, że trudno było do nich celować. Niemniej jednak, podczas gdy pierwsza próba z karabinów maszynowych była nieskuteczna ze względu na zasięg, druga seria strzałów była w stanie zabić „pewną liczbę” ptaków. Później tego samego dnia napotkano małe stado i zabito „prawdopodobnie tuzin” ptaków[5].
\end{frame}
\begin{frame}
„gdybyśmy mieli dywizję z odpornością tych ptaków na kule, to moglibyśmy walczyć przeciwko każdej armii na świecie. One (emu – przyp. aut.) mogą mierzyć się z karabinami maszynowymi z wrażliwością czołgów. Są jak Zulusi, których nawet pociski dum-dum nie mogły powstrzymać.”[8]
\end{frame}
\begin{frame}
\frametitle{Drugi okres}
Po wycofaniu się wojska, emu nadal dewastowały uprawy. Rolnicy ponownie więc poprosili o wsparcie, powołując się na upalną pogodę i suszę, która spowodowała, że tysiące ptaków zaatakowało ich farmy. James Mitchell, premier Australii Zachodniej, udzielił swojego poparcia aby ponowić pomoc wojskową. W tym samym czasie światło dzienne ujrzał raport, według którego w pierwszej fazie operacji zostało zabitych 300 emu[9].
\end{frame}
\begin{frame}
Ostatecznie 12 listopada minister obrony zatwierdził wznowienie działań wojskowych. Swoją decyzję bronił w Senacie, wyjaśniając dlaczego żołnierze byli dotąd niezbędni do walki z dużą populacją emu. Chociaż wojsko zgodziło się udzielić wsparcia rządowi Australii Zachodniej, w oczekiwaniu, aż ten ostatni zapewni potrzebnych ludzi, major Meredith ponownie został wysłany w teren, z powodu widocznego braku ludzi doświadczonych w obsłudze karabinów maszynowych[10].
\end{frame}
\begin{frame}
\frametitle{Echa akcji}
Pomimo problemów związanych z odławianiem ptaków, rolnicy w regionie ponownie zwracali się o pomoc wojskową w latach 1934, 1943 i 1948. Ale rząd za każdym razem odrzucał prośby osadników (w 1943 rolnicy domagali się nawet lotniczego bombardowania stad emu). Zamiast tak radykalnych posunięć, spopularyzowano jednak system nagród, zainicjowany jeszcze w 1923. Był on dość skutecznie kontynuowany, gdyż na przykład tylko w okresie sześciu miesięcy w 1934 odnotowano 57 034 zgłoszeń po nagrody za odłowione emu[12].
\end{frame}
\begin{frame}
Przez cały okres lat 30. i nieco później, prowadzono także ogradzanie terenów uprawnych, które stało się popularnym sposobem utrzymywania emu z daleka od obszarów rolniczych (ograniczyło to również napływ innych szkodników, takich jak dingo i króliki).
\end{frame}
\begin{frame}[allowframebreaks]
\frametitle{Bibliografia załącznikowa} 
https://pl.wikipedia.org/wiki/Wojna\_Emu

1 - Looking back: Australia’s Emu Wars. nationalgeographic.com.au. (ang.).

2 - Murray Johnson: Feathered foes: soldier settlers and Western Australia's 'Emu War' of 1932 „Journal of Australian Studies” (nr 88), s. 147.

3 - Johnson, s. 148.

4 - Johnson, s. 150.

5 - Machine Guns Sent Against Emu Pests, „The Argus” z 3 listopada 1932, s. 2; Johnson, s. 152.

6 - Johnson, s. 153.

7 - Johnson, s. 152.

8 - Johnson, s. 153; War on Emus, „The Argus” z 10 listopada 1932, s. 8.

9 - New Strategy In A War On The Emu, „Sunday Herald” z 5 lipca 1953, s. 13

10 - Emu War Again, „The Canberra Times” z 12 listopada 1932, s. 1.

11 - Emu War Again, s. 1.

12 - Johnson, s. 155.

13 - B. Crew: The Great Emu War: In which some large, flightless birds unwittingly foiled the Australian Army [w:] „Scientific American Blogs” z 4 sierpnia 2014; J.G. Gore: Looking Back: Australia's Emu Wars [w:] „Australian Geographic” z 2 listopada 2016
\end{frame}


\end{document}




